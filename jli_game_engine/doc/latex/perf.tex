\hypertarget{perf_perf_overview}{}\section{Overview}\label{perf_perf_overview}
This page discusses general performance issues related to assimp.\hypertarget{perf_perf_profile}{}\section{Profiling}\label{perf_perf_profile}
assimp has built-\/in support for {\itshape very} basic profiling and time measurement. To turn it on, set the {\ttfamily G\+L\+O\+B\+\_\+\+M\+E\+A\+S\+U\+R\+E\+\_\+\+T\+I\+M\+E} configuration switch to {\ttfamily true} (nonzero). Results are dumped to the log file, so you need to setup an appropriate logger implementation with at least one output stream first (see the \hyperlink{usage_logging}{Logging Page } for the details.).

Note that these measurements are based on a single run of the importer and each of the post processing steps, so a single result set is far away from being significant in a statistic sense. While precision can be improved by running the test multiple times, the low accuracy of the timings may render the results useless for smaller files.

A sample report looks like this (some unrelated log messages omitted, entries grouped for clarity)\+:

\begin{DoxyVerb}Debug, T5488: START `total`
Info,  T5488: Found a matching importer for this file format


Debug, T5488: START `import`
Info,  T5488: BlendModifier: Applied the `Subdivision` modifier to `OBMonkey`
Debug, T5488: END   `import`, dt= 3.516 s


Debug, T5488: START `preprocess`
Debug, T5488: END   `preprocess`, dt= 0.001 s
Info,  T5488: Entering post processing pipeline


Debug, T5488: START `postprocess`
Debug, T5488: RemoveRedundantMatsProcess begin
Debug, T5488: RemoveRedundantMatsProcess finished 
Debug, T5488: END   `postprocess`, dt= 0.001 s


Debug, T5488: START `postprocess`
Debug, T5488: TriangulateProcess begin
Info,  T5488: TriangulateProcess finished. All polygons have been triangulated.
Debug, T5488: END   `postprocess`, dt= 3.415 s


Debug, T5488: START `postprocess`
Debug, T5488: SortByPTypeProcess begin
Info,  T5488: Points: 0, Lines: 0, Triangles: 1, Polygons: 0 (Meshes, X = removed)
Debug, T5488: SortByPTypeProcess finished

Debug, T5488: START `postprocess`
Debug, T5488: JoinVerticesProcess begin
Debug, T5488: Mesh 0 (unnamed) | Verts in: 503808 out: 126345 | ~74.922
Info,  T5488: JoinVerticesProcess finished | Verts in: 503808 out: 126345 | ~74.9
Debug, T5488: END   `postprocess`, dt= 2.052 s

Debug, T5488: START `postprocess`
Debug, T5488: FlipWindingOrderProcess begin
Debug, T5488: FlipWindingOrderProcess finished
Debug, T5488: END   `postprocess`, dt= 0.006 s


Debug, T5488: START `postprocess`
Debug, T5488: LimitBoneWeightsProcess begin
Debug, T5488: LimitBoneWeightsProcess end
Debug, T5488: END   `postprocess`, dt= 0.001 s


Debug, T5488: START `postprocess`
Debug, T5488: ImproveCacheLocalityProcess begin
Debug, T5488: Mesh 0 | ACMR in: 0.851622 out: 0.718139 | ~15.7
Info,  T5488: Cache relevant are 1 meshes (251904 faces). Average output ACMR is 0.718139
Debug, T5488: ImproveCacheLocalityProcess finished. 
Debug, T5488: END   `postprocess`, dt= 1.903 s


Info,  T5488: Leaving post processing pipeline
Debug, T5488: END   `total`, dt= 11.269 s
\end{DoxyVerb}


In this particular example only one fourth of the total import time was spent on the actual importing, while the rest of the time got consumed by the \hyperlink{postprocess_8h_a64795260b95f5a4b3f3dc1be4f52e410a9c3de834f0307f31fa2b1b6d05dd592b}{ai\+Process\+\_\+\+Triangulate}, \hyperlink{postprocess_8h_a64795260b95f5a4b3f3dc1be4f52e410a444a6c9d8b63e6dc9e1e2e1edd3cbcd4}{ai\+Process\+\_\+\+Join\+Identical\+Vertices} and \hyperlink{postprocess_8h_a64795260b95f5a4b3f3dc1be4f52e410a16979c68f93d283c6886abf580d557b1}{ai\+Process\+\_\+\+Improve\+Cache\+Locality} postprocessing steps. A wise selection of postprocessing steps is therefore essential to getting good performance. Of course this depends on the individual requirements of your application, in many of the typical use cases of assimp performance won't matter (i.\+e. in an offline content pipeline). 
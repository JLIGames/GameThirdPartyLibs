Extracts one or more embedded texture images from models.

\subsubsection*{Syntax\+:}


\begin{DoxyCode}
\hyperlink{namespaceassimp}{assimp} extract <model> [<out>] [-t<n>] [-f<fmt>] [-ba] [-s] [common parameters]
\end{DoxyCode}


\subsubsection*{Parameters\+:}

{\ttfamily  model~\newline
}~\newline
 Required. Relative or absolute path to the input model. 

{\ttfamily  out~\newline
}~\newline
 Optional. Relative or absolute path to write the output images to. If the file name is omitted the output images are named {\ttfamily $<$model-\/filename$>$}~\newline
 The suffix {\ttfamily \+\_\+img$<$n$>$} is appended to the file name if the -\/s switch is not specified (where {\ttfamily $<$n$>$} is the zero-\/based index of the texture in the model file).~\newline


The output file format is determined from the given file extension. Supported formats are B\+M\+P and T\+G\+A. If the file format can't be determined, the value specified with the -\/f switch is taken. ~\newline
 Format settings are ignored for compressed embedded textures. They're always written in their native file format (e.\+g. jpg). 

{\ttfamily -\/t$<$n$>$~\newline
 }~\newline
 Optional. Specifies the (zero-\/based) index of the embedded texture to be extracted from the model. If this option is {\itshape not} specified all textures found are exported. The long form of this parameter is {\ttfamily --texture=$<$n$>$}. 

{\ttfamily -\/ba$<$n$>$~\newline
 }~\newline
 Optional. Specifies whether output B\+M\+Ps contain an alpha channel or not. The long form of this parameter is {\ttfamily --bmp-\/with-\/alpha=$<$n$>$}. 

{\ttfamily -\/f$<$n$>$~\newline
 }~\newline
 Optional. Specifies the output file format. Supported formats are B\+M\+P and T\+G\+A. The default value is B\+M\+P (if a full output filename is specified, the output file format is taken from its extension, not from here). The long form of this parameter is {\ttfamily --format=$<$n$>$}. 

{\ttfamily -\/s$<$n$>$~\newline
 }~\newline
 Optional. Prevents the tool from adding the {\ttfamily \+\_\+img$<$n$>$} suffix to all filenames. This option must be specified together with -\/t to ensure that just one image is written. The long form of this parameter is {\ttfamily --nosuffix}. 

{\ttfamily  common parameters~\newline
}~\newline
 Optional. Import configuration \& postprocessing. Most postprocessing-\/steps don't affect embedded texture images, configuring too much is probably senseless here. See the \hyperlink{common}{common parameters page } for more information. 





\subsubsection*{Sample\+:}


\begin{DoxyCode}
\hyperlink{namespaceassimp}{assimp} extract test.mdl test.bmp --texture=0 --validate-data-structure
\hyperlink{namespaceassimp}{assimp} extract test.mdl test.bmp -t=0 -vds
\end{DoxyCode}


Extracts the first embedded texture (if any) from test.\+mdl after validating the imported data structure and writes it to {\ttfamily test\+\_\+img0.\+bmp}.


\begin{DoxyCode}
\hyperlink{namespaceassimp}{assimp} extract files\(\backslash\)*.mdl *.bmp 
\hyperlink{namespaceassimp}{assimp} extract files\(\backslash\)*.mdl *.bmp 
\end{DoxyCode}


Extracts all embedded textures from all loadable .mdl files in the 'files' subdirectory and writes them to bitmaps which are named {\ttfamily $<$model-\/file$>$\+\_\+img$<$image-\/index$>$.bmp} 